\documentclass{report}
\usepackage[T1]{fontenc}
\usepackage{jaytex}
\renewcommand\chaptername{Lecture}

\begin{document}
\title{Notes on \textit{EE 219A: Numerical Simulation and Modeling} lectures by Professor Jaijeet Roychowdhury, Spring 2021}
\author{Jay Monga}
\date{Last updated \today}
\maketitle

\tableofcontents

\chapter{Introduction}
Website login info:
\begin{itemize}
    \item Username: 219Aspr2021
    \item Password: Python0Julia1?
\end{itemize}

\section{Course Coverage}
We will be covering
\begin{itemize}
    \item Models: Mainly circuits, but we will see how this applies to other case
    \item System equations that arise from connections of modeled circuit components. DAEs will be looked at a lot, differential-algebraic equations
    \item Biochemical reaction equations (we will see how this looks pretty close to circuits)
    \item Analysis of noise in systems
    \item Effects of parameter variability
    \item Steady state analysis
\end{itemize}
We will have a more structured format then a standard electronics of ME course. In particular, we have two independent steps:
\begin{itemize}
    \item Modeling physical systems as nonlinear differential equations
    \item Solving these differential equations numerically in different ways ("analyses")
\end{itemize}
The seperation of these two steps gives us a nicer view of the modeling process.
\section{Why is Simulation Useful?}

Motivating example - We might have a super complicated circuit with tons and tons of resistors and we want to determine some waveform with respect to some input. We may desire to do this analytically, but this may not always be feasible or possible. We may proceed by taking analytically simplifications, but this may also be hard and also lead us to oversimplify.\\

Numerical simulations let us gain insights into a system that may be hard to analytically solve or simplify. We can experiment at will with numerical examples and gain an understanding of the system that may allow us to do well guided analytic simplifications. \\

The study of modeling also provides some mathematical beauty and lets us identify connetions across many disciplines that may be otherwise hidden.

\section{First Look at Modeling through Circuits}
\subsection{Why Circuits First?}
\begin{itemize}
    \item It's an EE class
    \item Historically, modeling/simulation is the most well developed in the world of circuit simulation
    \begin{itemize}
        \item Starting with SPICE tools introduced in 70's, created at Berkeley
    \end{itemize}
    \item Later adapted to other domains (biology, mechanics, etc.)
    \begin{itemize}
        \item Even use circuits to describe things that have nothing to do with circuits,since circuits provide a nice, precise way to specifiy differential equations
    \end{itemize}
\end{itemize}
\subsection{EE Devices/Elements}
\subsubsection{Two Terminal Devices}
\begin{itemize}
    \item Two sides labeled by $p$ and $n$, stands for positive and negative but that doesn't actually have to be what the potential on the respective ends are
    \item Some voltage across across the device denoted by $V_{pn}$
    \item Current through the device denoted by $\vec{i_{pn}}$
    \item We have some equation for the relationship between voltage and current (typically voltage is the input) giving us a device model
    \begin{itemize}
        \item It's typically useful to plot out this relationship
    \end{itemize}
\end{itemize}
Let's look at some specific devices. We will start with some idealized models.
\begin{itemize}
    \item Linear resistor, model given by Ohm's Law: $i = \frac{V}{R}$
    \begin{itemize}
        \item We can see it has a perfect linear $I-V$ relationship
        \item The device is memoryless - current output at time $t$ only depends on values of voltage provided at time $t$
    \end{itemize}
    \item Linear capacitor, model given by $q = CV$
    \begin{itemize}
        \item We can take the derivative to get the $I-V$ relationship by taking the derivative of both sides to derive $i(t) = C \diff{V}{t}$
        \item Most idealized case is just two electric plates seperated by some insulator
        \item This device is not memoryless - the capacitor can be viewed as an integrator, which has some fundemental aspect of memory
    \end{itemize}
    \item Linear inductor, model given by $\phi = Li$
    \begin{itemize}
        \item We can take the derivative to get the $I-V$ relationship by taking the derivative of both sides to derive $V(t) = L \diff{i}{t}$
        \item We can also view the inductor as the dual of a capacitor
        \item Of course, this component will also not be memoryless - we can view it as a differentiator
    \end{itemize}
    \item Independent sources
    \begin{itemize}
        \item Can provide voltage and current
        \item Either compleletely flat vertical (voltage source) or horizontal (current source) line
        \begin{itemize}
            \item Thus, the voltage source $I-V$ relationship cannot be expressed as a function of $V$
        \end{itemize}
    \end{itemize}
    \item Linear dependent sources
    \begin{itemize}
        \item This is actually a 4 terminal device, but it is often abstracted to have two terminal
        \item The device will act as voltage or current source, with supply equal to some multiple of an input voltage or current
        \item Voltage or current through the device depends on load
        \item As a guiding principal, $p$ known values are needed to completely specify $p$ inputs and outputs of a system.
    \end{itemize}
\end{itemize}
Like previously stated, we covered idealized device models
\begin{itemize}
    \item Linear controlled sources are really unphysical
    \item R, L, C, etc. parameters are idealized (may be temperature dependent, device limits)
\end{itemize}
The takeaway is that all models are wrong, but some are useful. It is imporant to be able to know to what extent a model creates some approximation, or else you will be in a pickle.

\subsection{"System Properties" of Elements}
\begin{definition}[Memoryless]
    The output of a system at time $t$ is dependent only on the input at time $t$, not values of the input in the past or future.
\end{definition}
We will say that memoryless systems have outputs that are functions of the inputs (use $y(t) = f(x)$), or else the outputs are functionals of the inputs (use $y(t) = \chi \{x(t)\})$
\begin{definition}[Scaling]
    $x(t) \rightarrow y(t) \implies c \cdot x(t) \rightarrow c \cdot y(t)$
\end{definition}
\begin{definition}[Superpostion]
    $x_1(t) \rightarrow y_1(t), x_2(t) \rightarrow y_2(t) \implies x_1(t) + x_2(t) \rightarrow y_1(t) + y_2(t)$
\end{definition}
\begin{definition}
    A system is \textbf{linear} if it satisfies superpostion and scalaing.
\end{definition}
The quickest way to show linearity is to prove
\begin{equation*}
    x_1(t) \rightarrow y_1(t), x_2(t) \rightarrow y_2(t) \implies \alpha_1 \cdot x_1(t) + \alpha_2 \cdot x_2(t) \rightarrow \alpha_1 \cdot y_1(t) + \alpha_2 \cdot y_2(t)
\end{equation*}


\chapter{Systems Continued}


\section{Linearity}
% Missed proof of linearity for capacitor
% Missed proof of non-linearity for independent sources
% Missed explanation of affine system (indepependent source)
We concluded our discussion of linear systems for now and start getting into the interesting stuff.

\section{Nonlinear Elements}
\subsection{Nonlinear Resistors}
% Example with i = tanh(kv)
A nonlinear resistor might obey the $I-V$ relationship
\begin{equation*}
    i = f(v)
\end{equation*}
for some nonlinear function $f$.

It's important for this function $f$ to be well defined, as in it has a defined output for every input over the domain of interest. For good simulation, we should want $f$ to be continuous and infinitely %?
differentiable

Whenever we can express $i = f(v)$, we say we have a \textbf{voltage-controlled} element. As you can imagine, the reverse relationship $v = g(i)$ describes as \textbf{current-controlled element}.

Not all types of elements will be defined as voltage-controlled or current conrolled. We can sometimes have elements defined in an \textbf{implicit form} in which there may not exist a well-defined direct relationship between $v$ and $i$. For example, consider the (purely hypothetical) relationship
\begin{equation*}
    i^2 + v^2 = 1
\end{equation*}
The relationship gives us a lot of issues if we try to define this as something voltage-controlled and current-controlled, namely
\begin{itemize}
    \item For a given $v$ we cannot determined a unique $i$
    \item For a given $i$ we cannot determine a unique $v$
    \item There are values of $v$ and $i$ for which we do not have a defined output
\end{itemize}
Although this was just a toy example, we see real world examples in nonlinear elements

Why is this important? The major reason is that this all determines how we write our circuit equations. It can impact solving methods (we can solve voltage controlled sources with nodal analysis) and number of solutions as well.

\subsection{Ideal Diode Model}
% Put in a picture of the diode
The ideal diode model has the $I-V$ relationship
\begin{equation*}
    i_d = I_s (e^{\frac{V}{V_t}} - 1)
\end{equation*}
and is parameterized by
\begin{itemize}
    \item $I_s$, the saturation current $(A)$. Typically takes on values between $10^{-8}$ (breadboard) to $10^{-12}$ chip
    \item $V_t$, thermal voltage with equation \begin{equation*}
        \frac{kT}{q}
    \end{equation*}
    itself with parameters \begin{itemize}
        \item $k$, Boltzman constant % Jaijeet didn't remember this
        \item $q$, unit electric charge % missed this
        \item $T$, absolute temperature in Kelvin
    \end{itemize}
\end{itemize}
If we wanted to plot this $I-V$ curve, it would look like a common exponential. We idealize this diode by assuming that any negative current or very low is 0.
% put in a plot of the curve

We can see that at all points in the curve, $v \cdot i \geq 0$. This means that power is always being dissapted by our device, or alternatively, no power is being generated. It's a nice sanity check to see that this holds. For example. we can see that forgetting the $-1$ in our equation would make the negative $V$ region of our inputs generate power
% add plot of this

\subsection{Bipolar Junction Transistor (BJT) model: Ebers-Moll}
We consider an NPN Transistor (NPN stands for some physical properties of the device). The device has three termianls B, C, E, which stands for base, collector, and emitter.

% needs picture of device

Before we get into the model, we consider the conservation equations for this device. We have KVL
\begin{equation*}
    V_{BE} + V{BC} - V_{CE} = 0 %  I think I got this wrong
\end{equation*}
and KCL
\begin{equation*}
    i_B + i_C + i_E = 0
\end{equation*}

We will figure out output current of this device with respect to the input, namly a relationship
\begin{equation*}
    \begin{bmatrix}
        i_C \\
        i_E
    \end{bmatrix}
    = \vec{f}
    (\begin{bmatrix}
        V_{BE} \\
        V_{BC}
    \end{bmatrix})
\end{equation*}

% need to put in simplified equivalent ciruit drawing of the device

We can do KCL at each node to get some starting equations.

\begin{description}
    \item[Base Node] 
    \begin{equation*}
        i_B = (1 - \alpha_F)i_F + (1 - \alpha_R)i_R
    \end{equation*}
    \item[Collector Node] 
    \begin{equation*}
        i_C = \alpha_F i_F - i_R
    \end{equation*}
    Note we can get $i_F, i_R$ from our ideal diode equations.
    \begin{align*}
        i_F &= \mathrm{diode}(V_{BE}; I_{SF})
        i_R &= \mathrm{diode}(V_{BC}; I_{SR})
    \end{align*}

    Now we can explicitly write out a vector system % I'm too lazy to do this rn
    and make it like sort of linear by altering our input vector.
\end{description}

\subsubsection{Characteristic Curves}
Similarly to how we used plots to analyze our diode model, we can inspect the Characteristic curves of our BJT to analyze its properties. The convention is to plot $i_C$ over $V_{CE}$ for various fixed values of $V_{BE}$. We call the reigion where $i_C$ increases the \textbf{saturation region} and when $i_C$ remains constant the \textbf{forward active region}.

% put in characteristic curve plot

\subsection{Metal Oxide Semiconductor Field-Effect Transistor (MOSFET)}
\begin{itemize}
    \item Just like the BJT transistor, the MOSFET has NPN \& PNP types as well
    \item We will examine the Shichman-Hodges (great people who were at Bell Labs, Dave Hodges used to be at Berkeley) (S-H) model
\end{itemize} 
There are three terminals we look at, Gate (G), Drain (D), Source (S). As before, we can define voltages across these terminals and currents through them.
% put in picture of this
There are two cases we can consider when modeling this device. In the simple case when we have $i_G = 0$, we have
\begin{equation*}
    i_D = f_{N^+}() % finish this later
\end{equation*}

\chapter{Non-linear Devices Continued}
\section{Shichman-Hodges Model}
% Need to put in drawing of this device, NPN transistor, label terminals, currents, voltages
\begin{align*}
    i_G &= 0 \\
    i_D &= f_{Nt} (V_{GS}, V_{DS}) = 
    \begin{cases}
        0 & V_{GS} < V_T \text{ (OFF)}\\ 
        \frac{\beta}{2} (V_{GS} - V_T)^2 & V_R \leq V_{GS} < V_{DS} + V_T \text{ (Active/Saturation Region)}\\
        \beta [(V_{GS} - V_T) - \frac{V_{DS}}{2}]V_{DS} & V_{GS} \geq V_{DS} + V_T \text{ (Triode/"Linear" Region)}
    \end{cases}
\end{align*}
% Put in Jaijeet's physics drawing
Intuition: As voltage increases, we get more excess electrons which make the device behave more like a resistor. Need in depth physics knowledge to go farther from here.

($\beta$ is typically some constant near $10^{-2}$)

It may be helpful to write things in terms of $V_{GS} - V_T$.

% Put in labeled drawing of characteristic curves
On the characteristic curves, we hold $V_{GS}$ over a varying $V_{DS}$.

If we reverse the doping (PNP type) the device behaves in an inverse manner: We will be repelling electrons and creating holes.

With $V_{DS} < 0$, $S$ and $D$ will swap. We can label them $D', S'$ now

% show original transistor and draw on swapped labels. Also draw box with original and swapped labels, and label currents as well

\begin{align*}
    i_D 
    &= - i_D' \\
    &= - f_{N+}(V_{GS'}, V_{D'S'}) \\
    &= - f_{N+}(V_{GS} - V_{DS}, -V_{DS}) \\
    &= f_{N-}(V_{GS}, V_{DS})
\end{align*}

And we can create an abstraction now

\begin{equation*}
    i_D = 
    \begin{cases}
        f_{N+}(V_{GS}, V_{DS}) & V_{DS} \geq 0 \\
        f_{N-}(V_{GS}, V_{DS}) & V_{DS} < 0 \\
    \end{cases}
\end{equation*}
% Wait I think you should check if the case conditions are correct

What changes for the PNP case?
\begin{itemize}
    \item $V_{GS}$ replaced by $- V_{GS}$, $V_{DS}$ replaced by $- V_{DS}$ for putting in $f_N(\cdot, \cdot)$
    \item Output from $f(\cdot, \cdot)$ actually gives $i_S = - i_D$
\end{itemize}

Our current equation is then
\begin{equation*}
    i_D = f_P(V_{GS}, V_{DS}) = - f_N (-V_{GS}, -V_{DS})
\end{equation*}

\subsection{Recap}

\begin{itemize}
    \item Schichman-Hodes is a basic, 2 parameter model
    \item This rocked in the old days, but we can do much better now
    \item Modern models are much more complicated to complement their robustness
\end{itemize}

\section{Actual Device Models}
% Give an idea of what the code is
% Maybe some screenshots or pastings

Main takeaways:
\begin{itemize}
    \item A HUGE number of parameters. Hundreds of them! (about 400)
    \item Many many lines of code
    \item This isn't even the most modern model!
\end{itemize}

\section{Non-linear Capacitors \& Inductors}
\subsection{Non-linear Capacitor}
Recall the linear capacitor equation
\begin{equation*}
    q = Cv \land i = \diff{q}{t} \implies i = c \diff{v}{t}
\end{equation*}
Let's introduce some nonlinear relationships
\begin{description}
    \item[Voltage-controlled nonlinear capacitor] 
\begin{equation*}
    q = f(v) \and i = \diff{q}{t}  \implies i = f'(v) \diff{v}{t}
\end{equation*}
    \item[Charge-controlled capacitor]
    \begin{equation*}
        V = g(q)
    \end{equation*}
    \item[Implicit capacitor]
    \begin{equation*}
        h(v, q) = 0
    \end{equation*}
\end{description}
\subsection{Non-linear Inductor}
\begin{description}
    \item[]
    % I missed all of these ...
\end{description}

\begin{example}[Depletion Capacitance]
    % Picture of circuit
    \begin{equation*}
        Q_{\mathrm{depl}}(v) = 
        \begin{cases}
            A C_{JO} \frac{\phi}{1 - m}(1 - (1 - \frac{n}{\phi})^{1 - m}) & v \leq \int_c \phi \\
            \text{complicated expression} & \text{otherwise}
        \end{cases}
        % Check that you got this down correctly
    \end{equation*}
    Check out all the parameters!

    % Put in plot
\end{example}

\begin{example}[Non-linear Inductor]
    % Put in picture of plot
    Here we can see a piecewise-linear interpolation which forms a model of a transformer magnetizing curve.
\end{example}

\section{Continuity, Derivatives, Smoothness}
% Include image of evaluating a derivative on an I-V curve at some voltage V*

% I may have missed Jaijeet explaining some terms

\begin{definition}
    A function is \textbf{smooth} when it and \textbf{all} it's derivatives (higher-order \& mixed) exist \& are well-defined
\end{definition}

% put in delta-epsilon definition of continuity

% put in examples of smooth, non-smooth functions

Why is this importance?
\begin{itemize}
    \item Numerical algorithms (ie. Newton-Raphson) use functions/derivatives
    \begin{itemize}
        \item N-R convergence is sensitive to derivatives
        \begin{itemize}
            \item non-smoothness breaks theoretical (and in practice) assurance of N-R convergence
        \end{itemize}
    \end{itemize}
    \item Physical systems typically smooth at fine enough resolution
    \begin{itemize}
        \item Non-smoothness indicative of unphysical model (take this with a grain of ault, just a sanity check)
        \begin{itemize}
            \item Incorrect modeling of underlying physics/math
            \begin{itemize}
                \item ex. Modeling a real device by giving it discrete properties
                \item Should be a continuous model, no diff eqs
                \item Careful with finite-state machines
            \end{itemize}
        \end{itemize}
    \end{itemize}
    \item Derivatives often have important physical connotations
    \begin{itemize}
        \item Ex. give us some idea about resistance of the device
    \end{itemize}
    \item 2\ts{nd} derivatives (Hessians) help us do optimization
    \item higher order derivatives can be important in some modeling situations
    \begin{itemize}
        \item distortion and intermodulation effects
    \end{itemize}
\end{itemize}

\end{document}
