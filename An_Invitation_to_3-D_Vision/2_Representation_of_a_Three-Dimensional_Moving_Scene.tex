\documentclass{book}
\usepackage[T1]{fontenc}

\usepackage{jaytex}
\begin{document}
\title{Notes on \textit{An Invitation to 3-D Vision} by Ma, Soatto, Kosecka, and Sastry}
\author{Jay Monga}
\date{\today}
\maketitle

\tableofcontents

\chapter{Introduction}

\section{Visual perception from 2-D images to 3-D models}

\section{A mathematical approach}

\section{A historical perspective}

\part{Introductory Material}

\chapter{Representation of a Three-Dimensional Moving Scene}

In this chapter, we focus on the first fundamental set of transformations central to the geometry of 3D-vision: \textit{Euclidean motion}, or \textit{rigid-body motion}.
This is pretty important for modeling how a camera moves, and where things are in a 3D scene.
Rigid-body motion is also super important for robotics in general, so it's a good idea to get a grasp on this!
We will start by going into what a Euclidean space is and then spend the rest of the time talking about rigid-body motions.

\section{Three-dimensional Euclidean space}
We are going to use $\mathbb{E}^3$ to denote three-dimensional Euclidean space.

\section{Rigid-body motion}

\section{Rotational motion and its representations}

\subsection{Orthogonal matrix representation of rotations}

\subsection{Canonical exponential coordinates for rotations}

\section{Rigid-body motion and its representations}

\subsection{Homogenous representation}

\subsection{Canonical exponential coordinates for rigid-body motions}

\section{Coordinate and velocity transformations}

\section{Summary}

\section{Exercides}

\section{Quaternions and Euler angles for rotations}

\end{document}