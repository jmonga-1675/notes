\documentclass{book}
\usepackage[T1]{fontenc}

\usepackage{jaytex}
\begin{document}
\title{Notes on \textit{An Invitation to 3-D Vision} by Ma, Soatto, Kosecka, and Sastry}
\author{Jay Monga}
\date{\today}
\maketitle

\tableofcontents

\chapter{Introduction}

\section{Visual perception from 2-D images to 3-D models}

\section{A mathematical approach}

\section{A historical perspective}

\part{Introductory Material}

\chapter{Representation of a Three-Dimensional Moving Scene}

In this chapter, we focus on the first fundamental set of transformations central to the geometry of 3D-vision: \textit{Euclidean motion}, or \textit{rigid-body motion}.
This is pretty important for modeling how a camera moves, and describing general motion in 3D space.
You may imagine how this is pretty important for robotics in general, as robots are often moving around in 3D space and we may want to model that, so its a good idea to have a grasp on rigid-body motion!
We will start by going into what a Euclidean space is and then spend the rest of the time talking about rigid-body motions.

\section{Three-dimensional Euclidean space}
A Euclidean space is almost like a vector space, except it doesn't contain vectors outright.
Instead, a Euclidean space is a set of real points, all forming an affine space over the reals.
We are going to use $\mathbb{E}^n$ to denote the $n$-dimensional Euclidean space, and thus $\mathbb{E}^3$ to denote the three-dimensional Euclidean space which we will restrict our focus to for the rest of this chapter.
A point $p \in \mathbb{E}^3$ corrosponds to a vector $\mathbf{X} \in \mathbb{R}^3$.
\begin{equation*}
    \mathbf{X} \dot{=}
    \begin{bmatrix}
        X_1 \\
        X_2 \\
        X_3 \\
    \end{bmatrix}
    = 
    \begin{bmatrix}
        X \\
        Y \\
        Z \\
    \end{bmatrix}
\end{equation*}
The $\mathbb{R}^3$ vector corrosponding to an $\mathbb{E}^3$ point can be called its coordinates.
Every point in $\mathbb{E}^3$ has unique coordinates in $\mathbb{R}^3$ and vice versa, forming a bijection!\\
The notion of points is important for defining what it means to have a vector in $\mathbb{E}^3$.

\theoremstyle{definition}
\begin{definition}[Vector]
    A vector $v$ in $\mathbb{E}^3$ is determined by a pair of points $p,q \in \mathbb{E}^3$ as a directed arrow $v = \vec{pq}$ connecting the two points.
\end{definition}



\section{Rigid-body motion}

\section{Rotational motion and its representations}

\subsection{Orthogonal matrix representation of rotations}

\subsection{Canonical exponential coordinates for rotations}

\section{Rigid-body motion and its representations}

\subsection{Homogenous representation}

\subsection{Canonical exponential coordinates for rigid-body motions}

\section{Coordinate and velocity transformations}

\section{Summary}

\section{Exercides}

\section{Quaternions and Euler angles for rotations}

\end{document}