\documentclass{report}
\usepackage[T1]{fontenc}
\usepackage{jaytex}
\renewcommand\chaptername{Lecture}

\begin{document}
\title{Notes on \textit{Math H113: Abstract Algebra} lectures by Professor Edward Frenkel, Spring 2021}
\author{Jay Monga}
\date{\today}
\maketitle

\tableofcontents

\chapter{Introduction}
Welcome to Math H113: Honors Abstract Algebra!
As opposed to Math 113: Abstract Algbra, H113 will proceed at a faster pace and cover more material.
As a result, some material like the details of a proof or definition will be covered just enough to pique our curiousity but rich coverage left aside for independent study.
In these notes, I will try to fill in some gaps of the coverage, and we'll see how this goes.

\section{First Look at Sets}
We start by providing a definition of algebra, the focus of this class.

\theoremstyle{definition}
\begin{definition}
    \textbf{Algebra} is the study of sets with operations satisfying certain axioms.
\end{definition}

But what even is a set?
We can think of a set as just a colection of items.
The idea seems simple enough, but leaving it this simple can cause us problems later down the road, as we will see.
For now, we will leave it as that.
We can see how the set is one of the basic building blocks of mathematics, and we can start to explore some interesting behavior that comes from defining special structure and operations on elements of a set.
This leads to our discussion of groups.

\section{Groups}
Groups are the fundamental set in this class.
\theoremstyle{definition}
\begin{definition}
A \textbf{group} is the set of symmetries for a given object.
\end{definition}
Consider a circle.
We can define a group for this object by the set of all rotations of that circle by an angle modulo 2 $\pi$. 
\begin{figure}
    \caption{Insert circle diagram here}
\end{figure}
Notice that applying two rotations is the same as applying the composition of the rotations, the 0 degree rotation composes with a rotation to produce the same rotation, and every rotation has an inverse rotation that will reset it to the 0 degree rotation.
These three properties are what makes the set of all 2D rotations $SO(2)$ a group. \\
Let's now formalize our definition of a group

\theoremstyle{definition}
\begin{definition}
A \textbf{group} is the set $G$ with an operation satisfying axioms of
\begin{enumerate}
    \item $\forall x, y z \in G, x \cdot (y \cdot g) = (x \cdot y) \cdot g \text{ (Associativity)}$
    \item $\exists e \in G \text{ s.t. } e \cdot g = g \cdot e = g, \forall g \in G \text{ (Identity element)}$
    \item $\forall g \in G, \exists g' \text{ s.t. } g \cdot g' = g' \cdot g = e \text{ (Inverse element)}$
\end{enumerate}
\end{definition}

Now we can go through some examples!
We already introduced $SO(2)$, which Prof. Frenkel brings up in the beginning of \textit{Love and Math}, but there are other cool examples as well.

\subsection{Galois Group}
First, let's introduce the concept of a field.
\theoremstyle{definition}
\begin{definition}
A \textbf{field} is a set defined with two binary addition and multiplication operations that satisfy a set of axioms.
\end{definition}
We will go more in detail later; these axioms are similar to the axioms of a vector space.
Common fields include $\mathbb{Q}, \mathbb{R}, \mathbb{C}$. \\
The Galois group can be seen as extending a field into a set of symmetries.\\
The Galois group is named after its discoverer \href{https://en.wikipedia.org/wiki/%C3%89variste_Galois}{Évariste Galois} who tragically died in a duel at a young age.
One interesting application of his work was for determining existence of solutions in terms of a radical for polynomials with degree greater than or equal to 5.
This problem was significantly studied by the great mathematician \href{https://en.wikipedia.org/wiki/Muhammad_ibn_Musa_al-Khwarizmi}{Muhammad ibn Mūsā al-Khwārizmī} 1000 years before Galois's time.\\
The word algorithm is derived from al-Khwārizmī and algebra, the topic of this class, comes from a book he wrote called al-jabr.

\subsection{Braid Group}

\section{Algebra vs Analysis}

\section{Paradoxes}

\chapter{Set Theory}
\section{Recap}
Last lecture, we talked about
\begin{itemize}
    \item Notion of a set
    \item Russel Paradox ($R = \{x|x \notin x\}$)
    \item $ZFC$ axioms to solve carelessness in definition of a set that lead to logical inconsistencies like Russel Paradox
    \item It's hard to precisely define a set is, so to some extent we have to take some faith in that we understand them to be some agreeable object
\end{itemize}
As always, use Wikipedia or nLab to look up things you are interested in.

\section{Sets Continued}
As before, we look at sets as a collection of objects. Let's look at types of sets
\begin{itemize}
    \item empty set: this is a unique set with no items denoted by $\emptyset$
    \item finite sets: sets with a finitely many elements
    \item functions: a specific type of relation between two sets, but can be seen as a set itself
    \item subset: a set that is a collection of some elements of another set (the integers are a subset of the real numbers)
\end{itemize}

\theoremstyle{definition}
\begin{definition}
    Suppose $A$ and $B$ are two sets. A \textbf{function} $f: A \rightarrow B$ is a rule that assigns to every every element $a \in A$ an element (one and only one) of B, denoted $f(a)$
\end{definition}
\begin{definition}
    A function $f: A \rightarrow B$ has \textbf{domain} $A$ and \textbf{codomain} $B$
\end{definition}
\subsection{Functions}
Functions are pretty important to us, so we will dive deeper into them.
\begin{example}
    In single-variable calculus, we consider functions $f: \mathbb{R} \rightarrow \mathbb{R}$
\end{example}
\begin{example}
    Consider $A = \{a, b, c\}, B = \{x, y, z, w\}, f:A \rightarrow B$
    \begin{itemize}
        \item A valid $f$ would be
        \begin{itemize}
            \item $f(a) = x$
            \item $f(b) = y$
            \item $f(c) = z$
        \end{itemize}
        \item An invalid $f$ would \begin{enumerate}
            \item $f(a) = x$
            \item $f(b) = y$
            \item $f(c) = z$
            \item $f(c) = w$
        \end{enumerate}
        since we see that $c$ is getting mapped to two different elements $z, w$.
    \end{itemize}
    Function properties to be familiar with
    \begin{itemize}
        \item injective, or one-to-one
        \item surjective, or onto
        \item bijective, one-to-one and onto
        \item uniquness of inverse (prove through some algebra looking at $g' \circ f \circ g $)
    \end{itemize}
    Remember to read function composition from right to left, ie applying $f$ first and then $g$ is written as $g \circ f$.
\end{example}
\section{Cardinality}
With a notion of functions and their properties, we can start talking about set sizes.
\begin{definition}
    A set $A$ is \textbf{finite} if there exists a bijection with 
\end{definition}

\chapter{Groups}

\section{Binary Operations}
% I missed some aside about cartesian products

\begin{definition}
    A \textbf{binary operation} on a set $S$ is a function
    \begin{equation*}
        f: S \times S \rightarrow S
    \end{equation*}
    We will denote $f(a, b)$ as $a * b$ when it is clear what $f$ is, or we are talking about some general $f$.
\end{definition}

\begin{definition}
    $*$ is called \textbf{commutative} if $\forall a,b \in S, a * b = b* a$
\end{definition}


\begin{definition}
    $*$ is called \textbf{associative} if \begin{equation*}
        (a * b) * c = a * (b * c)
    \end{equation*}
\end{definition}

\begin{remark}
    If $*$ is associative, then $*$ defines an $n$-ary operation (function on $n$ ordered arguments).
\end{remark}

Binary operations in general are not commutative or associative, but we can see that they give rise to interesting structure if they do hold.

% I missed some things Frenkel was saying about importance of associativity

\begin{example}
    Consider the binary operation of $+$ on $\mathbb{N}$. We can trivially see that
    \begin{itemize}
        \item $+$ is commutative
        \item $+$ is associative
    \end{itemize}
\end{example}

\begin{example}
    Again consider $+$, but on $\mathbb{Z}^+$. We notice that the element $0$ has a special property
    \begin{equation*}
        \forall a \in \mathbb{Z}^+, a + 0 = 0 + a = a
    \end{equation*}
\end{example}

\begin{definition}
    An element $e$ of a set $S$ with a defined binary operation $*$ is called the \textbf{identity element} of $S$ if
    \begin{equation*}
        \forall a \in S, a * e = e * a = a
    \end{equation*}
\end{definition}

\begin{lemma}
    For binary structure $(S, *)$, an identity element in $S$, if it exists, must be unique.
\end{lemma}

\begin{proof}
    Uniquness proofs often follow the structure of assuming two seperate entities exist, and then proving that they must be the same entity. In this case, we assume we have two elements $e, e'$ that satisfy the definition of identity element on binary structure $(S, *)$. 

    Let's consider the value of $e * e$. using the identity property of $e$, we have \begin{equation*}
        e * e' = e'
    \end{equation*}
    But using the identity property of $e'$, we must have
    \begin{equation*}
        e * e' = e
    \end{equation*}
    These two equalities give us $e = e'$.

    We can put this proof in one cute line as well
    \begin{equation*}
        e' = e * e' = e
    \end{equation*}
    But be sure to explain your steps!
    \qed
\end{proof}

\begin{example}
    Consider the set of counterclockwise rotations by some angle $\varphi$.
    % I missed something he was talking about
    We define $+$ on this set as
    \begin{equation*}
        x + y \, \bmod 2 \pi
    \end{equation*}
\end{example}

\begin{example}
    On $\mathbb{Z}, \mathbb{Q}, \mathbb{R}$ consider the $+$ operation with a special rule: for $C \in \mathbb{R}^+$, $x + y$ will take on the value of its remainder when divided by $C$. This is called addition modulo $C$. We can say $x \equiv y \, \bmod C$ if $\exists n \in \mathbb{Z}$ s.t. $x = y + n C$
\end{example}

% definition of equivalence relation, explain how mod is one
% I missed some things about partitions, equivalence classes, and residue classes
With our equivalence class established, we can say that
\begin{equation*}
    \tilde{x} \sim x', \tilde{y} \sim y' \implies \tilde{x} * \tilde{y} \sim x' + y'
\end{equation*}
% need to clarify if this is just for mod or in general

\begin{example}
    Consider the set of rotations of a sphere (known as $SO(3)$). We can consider each element of this set as a point on the sphere. In general, the operation of composing two rotations will not be commutative, but associative. This binary operation will also have an identity element.
\end{example}

\begin{example}
    We will show that complex numbers give us an alternate, but equivalent, description of the symmetries of a circle. Recall the standard definitions of addition and multiplication for complex numbers. 
    % Define addition and multiplication for complex numbers

    \begin{remark}
        $\mathbb{N}, \mathbb{Z}^+, \mathbb{Z}, \mathbb{Q}, \mathbb{R}$ have 2 natural operations: $+$ and $\cdot$
    \end{remark}
\end{example}

\chapter{Groups cont.}
\section{Complex Numbers}
% Lol you might want to rewatch this part of lecture
\begin{itemize}
    \item Complex numbers are commutative group under multiplication and addition
    \begin{itemize}
        \item This property is sort of unique to groups isomorphic to $\mathbb{R}^2$
    \end{itemize}
    \item When restricted to unit circle, group of complex numbers under multiplication isomorphic to residue classes of $\mathbb{Z}$ modulo $2 \pi$
    \begin{itemize}
        \item Eulers formula
        \begin{itemize}
            \item Algebra vs analysis discussion
            \item Taylor series derivation
            \item Another proof I missed I think
        \end{itemize}
    \end{itemize}
\end{itemize}

\section{Formalized Groups}
\begin{definition}
    % Clean this up to use correct group axioms
    A \textbf{group} is a set $S$ with an associative binary operation $*$ such that
        \begin{description}
            \item[(Closed under $*$)] $*: S \times S \rightarrow S$
            \item[(Identity element)] $\exists e \in S$ s.t. $\forall x \in S,\, e * x = x * e = x$
            \item[(Inverse element)] $\forall x \in S, \, \exists x' \in S$ s.t. $x * x' = x' * x = e$
        \end{description}
\end{definition}
% Frenkel gave some examples that I missed

\begin{example}
    Consider $M_{n \times n}$, the set of all $n \times n$ matrices with entires in $\mathbb{R}$. Is this a group under the binary operation of matrix multiplication?
    \begin{itemize}
        \item Axiom 1 holds since an $n \times n$ matrix times an $n \times n$ matrix
    \end{itemize}

    However, consider
    \begin{equation*}
        GL_n\{\mathbb{R}\} = \{A \in M_{n \times n}(\mathbb{R}) \,|\, \det A \neq 0\}
    \end{equation*}
    which is the set of all invertible $n \times n$ matrices. This set will form a group.
\end{example}

It's now time to define the notion of a subgroup

\begin{definition}
    A \textbf{subgroup} of a group $G$ is a subset of $G$ that also forms a group
\end{definition}

\begin{remark}
    It's not enough for a subset of $G$ to be closed under $*$ to be considered a subgroup, we need to check the other properties too. Conside $\mathbb{Z}^+$
    % finish this discussion
\end{remark}

% Investigate upper-half plane of complex numbers
% Frenkel goes through some weird group SL_2 (R) under this mobius transformation thing
% and then apparently SL_2(Z) is an interesting subgroup of that
% Also look up modular group, modular forms

\begin{definition}
    A \textbf{homomorphism} $h: G^{*_G} \rightarrow H^{*_H}$ is a function between two sets such that
    \begin{equation*}
        h(g_1 *_G g_2) - h(g_1) *_H h(g_2)
    \end{equation*}
\end{definition}

\begin{example}
    We just talked about a homomorphism from $G = SL_2 (\mathbb{R})$ to $H_+$ (upper half plane of complex numbers)
\end{example}

\subsection{Important Results}
Assume we have some group structure $(G, *)$
\begin{theorem}
    \begin{equation*}
        a * b = a * c \implies b = c
    \end{equation*}
\end{theorem}
\begin{proof}
    We are guarenteed to have an inverse for $a$
\end{proof}

\begin{theorem}
    The equation
    \begin{equation*}
        a * x = b
    \end{equation*}
    has a unique solution
    \begin{equation*}
        x  = a' * b
    \end{equation*}
\end{theorem}
\begin{proof}
    % standard uniquess of solution proof
\end{proof}
% \begin{corollary}
%     $\forall a \in G, a'$ is unique
% \end{corollary}
% need to finish this 

\end{document}