\documentclass{report}
\usepackage[T1]{fontenc}
\usepackage{jaytex}
\renewcommand\chaptername{Lecture}

\begin{document}
\title{Notes on \textit{EE 290-005: Integrated Perception, Learning, and Control} lectures by Professors Yi Ma, Jitendra Malik, Shankar Sastry, and Claire Tomlin, Spring 2021}
\author{Jay Monga}
\date{Last updated \today}
\maketitle

\tableofcontents

\chapter{Introduction}
Welcome to EE 290, which can be seen as an advanced robotics course.
The faculty has a diverse set of specializations which cover the main themes of the course: perception, planning and control.
This is the first time the course is being offered, so it's going to be a fun ride.
\paragraph{Goal:} Study the integrated roles of perception, learning, and control in a closed-loop for autonomous robotic systems, under various levels of modeling uncertainty for the environment and of resource constraint for the agent.\\
Many robotics curriculums focus on small parts instead of the integration of perception, learning, and control systems, so this course will focus on the whole, closed-loop autonomous systems.\\

\section{Closed-Loop Autonomous System}

\paragraph{Model Uncertainty:}
\begin{itemize}
    \item deterministic
    \item stochastic
\end{itemize}

\paragraph{Resources:} Are we a private researcher at Google or a poor professor at Berkeley?
\begin{itemize}
    \item cost
    \item hardware
\end{itemize}

\section{Presentation and Project Suggestions}
We don't want paper presentations to be a waste of your and our time. Presentations should be engaging and educatiational. Make sure to cover
\begin{itemize}
    \item Problem formulation and assumptions (what is actually being solved? how is uncertainty modeled?)
    \item Justification for the proposed method (why not optimal control, learning, etc.)
    \item Generalizability in the proposed solution (is the solution general? or was a special case solved? how well?)
    \item Data and computational resourses needed (how easily can these methods be applied? are the results practical?)
\end{itemize}

In a similar manner, keep these tips in mind to get the most out of your projects.
\begin{itemize}
    \item Don't stay "cocooned" in your own specific field (ex. I only like vision). This class is about integration.
    \item Keep physical-world implementation in mind; this makes the work practical.
    \item Again, integration. Try to combine model-based and data-driven approaches.
    \item Think out-of-the box to create projects that really capture interest and imagination.
\end{itemize}

\section{History of Robotics}
Prof. Sastry likes to always start off with a history of robotics. 
\begin{itemize}
    \item "it's important to know whoose shoulders you are standing on"
    \item Prosthetics was one of the earliest examples of robotics
    \item Mechatronics became more advanced and even programmable.
    \item Development of cars and planes provided lots of advancement
    \item Term "ROBOT" coined in by Karel Capek called "RUR"
    \item Nyquist, Bode, and Turing driving lots of fundamental theory
    \item Norbert Wiener did foundational work on cybernetics
    \item Rudolf Kalman did fundamental work on controls
    \item Early telemanipulation work done at Argonne national lab
\end{itemize}

\section{Integrated Perception and Control}
Prof. Malik likes to start out with evolutionary biology and how developed our sensors, acuators, and CPUs.
\begin{itemize}
    \item Animals developed sight and movement around the same time
    \begin{itemize}
        \item "Gibson: We see in order to move, and we move in order to see"
    \end{itemize}
    \item Once we got locomotion down, we started to develop hands
    \item Use of hands to make tools correlated with growth of intelect
    \item While vision sensors today are as good as and better than our eyes, nothing really matches our hands yet
    \item Evolutionary progression came in order of vision and locomotion, manipulation, and then language.
    \begin{itemize}
        \item Success in AI seems to also come in this order
    \end{itemize}
    \item Prof. Malik has special intrest in general purpose robots that can work in homes
\end{itemize}

One field of locomition that has been studied a lot is gaits. Boston Dynamics has developed robots that can emulate animal gates very well, although more complex gaits in real life are left to be effectively realized. In essence, gaits do not encompass all forms of legged locomotion.

\paragraph{Coupled Perception and Action}
Vision allows a way for a robot to have enough knowledge of its environment to in theory plan robust motion.\\

Visual landmarks are quite useful for planning in an environment.
One classical approach is SLAM - develop a complete map, and then plan in it.
However, it may better to utilize visual landmarks rather doing full scene reconstruction. \\

"Instead of trying to produce a programme to simulate the adult mind, why not try to simulate the child?" - Turing

\paragraph{Six Lessons} What we learned about learning by observing people
\begin{itemize}
    \item 
\end{itemize}

Vision can help us learn from actions and develop a environment model and policy. We also want to incorporate perception in the control loop (Prof. Tomlin's work).

\chapter{Learning and Vision}


\section{3D Perception by Humans}

Humans rely on many visual cues that provide information about the environment

% Need to explain what these are
\begin{itemize}
    \item Accomodation
    \begin{itemize}
        \item Change eye focal length
    \end{itemize}
    \item Convergence
    \item Convergence angle vs distance
    \item Motion (optical flow)
    \begin{itemize}
        \item Estimate derivative
        \item Most important signal for depth
        \item Can use it to segment occluded objects
    \end{itemize}
    \item Binocular Stereopsis
    \begin{itemize}
        \item Stereo vision with our eyes
    \end{itemize}
    \item Pictorial Cues
    \begin{itemize}
        \item We can percieve surface normals, not just depth, through a combination of various pictoral cues
    \end{itemize}
    \item Curvilinear Grouping
    \begin{itemize}
        \item We can assume depth order since boundaries are smooth in nature
        % Add the pictures
    \end{itemize}
    % Jitendra skipped a lot
    \item Texture gradients
    \begin{itemize}
        \item Different objects appear different in different distances; surface normals get modified
    \end{itemize}
    \item Shading
    \begin{itemize}
        \item Changes in shade can make us percieve depth
        % Jitendra had some cool work here
    \end{itemize}
    % and more...
\end{itemize}

    The main point is that we can't effectively use these features by default; we need to learn how to do so.

    % Lol I missed the name of this section
    \section{}
    Jitendra likes to emphasize the 3 R's of Vision
    \begin{itemize}
        \item Recognition
        \begin{itemize}
            \item Classification and categorization of objects in the scene
            \item Determining properties of objects as well (density, best grasping points)
        \end{itemize}
        \item Reorganization
        \begin{itemize}
            \item Making sense of the objects we know are there % I think? I kind of missed this part
        \end{itemize}
        \item Reconstruction
        \begin{itemize}
            \item Recovering 3D scene from image
            \item The inverse graphics problem (not just depth!)
        \end{itemize}
    \end{itemize}
    People used to view all these We should view these three processes as interacting and each having their own feedback loops.

    Deep learning has done a lot of nice work connecting recognition and reorganization
    \begin{itemize}
        \item Determine bounding box, mask, and label
        % The one paper he used pictures from
    \end{itemize}

    Where do we go from here? 3D reconstruction still needs a lot of work.

    \begin{itemize}
        \item Classical work in 3D vision has given us some nice reconstruction
        % Explain these things
        \begin{itemize}
            \item Blocks world
            \item Morphable models
            \item Shape from shading
            \item Structure from motion
        \end{itemize}
        \item Now deep learning gives us some nice things
        \begin{itemize}
            \item Mesh R-CNN by Jitendra's gropu
            \begin{itemize}
                \item Detect object
                \item Guess voxels
            \end{itemize}
        \end{itemize}
    \end{itemize}

    "Vision can be solved because we do it all the time" - Jitendra

    \section{Robotics and Vision}

    "Robotics is the killer app for vision" - Yi

    \begin{itemize}
        \item Yi's first project as a PhD student was for self-driving
        \item In his MASKS book, talks about vision for cars and helicopters
        \item Need to make vision computational and mathematical
    \end{itemize}

    We will kind of be summarizing the MASKS book

    % Put these in seperate subsections
    \begin{itemize}
        \item Images and geometry
        \begin{itemize}
            \item First computer vision pepople were artists
            \item Perspective projection ideas
            \item A lot happened within the turn of the last century!
            \item Ma comes in to help give unifying mathematical perspectie
            \item Now we think about not using so many views, because even a single image gives us a lot of information
        \end{itemize}
        \item Geometry for multiple images
        % Probably want to add actual equations
        \begin{itemize}
            \item "hat" matrix for cross product
            \item Homogenous coordinates
            \item Perspective projection equation
            \item Homogenous representation of 3D line
            \item Projection of line onto image is intersection of triangle plane and image plane
            \item Specify plane with normal
            \item Incidence relation from multiple cameras is the sufficient and necessary constraint for 3D reconstruction
            \item We can construct matrices whose rank gives us our ability to reconstruct
            \item Special case when points lie on the same plane
            \item Iterated solving of depth and perspective
        \end{itemize}
        \item Geometry for single images
        \begin{itemize}
            \item Utilize symmetry in scene
            \item Windows, walls, edges are straight/orthogonal features
            \item Symmetry lets us infer the scene at multiple views
            \item Group defintion of symmetric scene % I missed this
            \item Let's us build prceise geometry from few images
        \end{itemize}
    \end{itemize}

    We kind of assumed that feature matching or initial views was given - this is where the other R's come in

    % Learning paper, i forgot the name that Yi put out
    \begin{itemize}
        \item Traditional 3D reconstruction bypasses reconstruction problem
        \begin{itemize}
            \item Suffer from lack of features
            \item textureless objects
            \item reflection
            \item representation patterns (fence)
            \item medium/large baseline
            \item moving objects
        \end{itemize}
        \item Deep learning seems to work work well for reconstruction tasks but...
        \item they are really only doing classification instead of reconstruction, not actually doing a great job
        \item Problem: we aren't utilizing priors about structure in a scene
        \item We actually don't care about depth of reconstructing every point in the image
        \item This can make the problem well defined
        \item Get wireframe from single image - this gives us good structure of the scene because we used priors and didn't try to get everything
        \item Learning helps us go from canny edge to wireframe
        \item Learning to find vanishing points
        % Rely on video for the rest, i'm getting tired
        \item Train netural network to identify plane of symmetry and use that to get correspondances
        \item Datasets are super important for AI projects, but it's kind of lacking for 3D scene
        \item HoliCity project
    \end{itemize}

\end{document}