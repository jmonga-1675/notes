\documentclass{report}
\usepackage[T1]{fontenc}
\usepackage{jaytex}
\renewcommand\chaptername{Lecture}

\begin{document}

\title{Lecture notes for \textit{EE 123: Digital Signal Processing} lectured by Kannan Ramchandran, Spring 2021}
\author{Jay Monga}
\maketitle

\tableofcontents

\chapter{Introduction}
Digital signal processing is a broad field that has existed for quite some time but remains hugely important to date.
Prof. Lustig likes to call it the "swiss-army knife" for scientists and engineers, and also says you can do a lot of badass things with DSP like steal cars. \\
This course will extend off of EE 120 to cover many aspects of digital signal processing, in both theory and implementation.
There will be a significant lab and project component of this course, although we won't be doing any HAM radio things as usual.

\section{Review of Common Signals}
We will first talk about common signals you should be familiar with.
But before then, let's make sure we remember the difference between continuous time, discrete time, and digital signals.
\theoremstyle{definition}
\begin{definition}
    \textbf{Continuous-time signals} are functions of a continuous independent variable. These will be written as $x(t)$.
\end{definition}
\begin{definition}
    \textbf{Discrete-time signals} are sequences of numbers. These will be written as $x[n]$.
\end{definition}
\begin{definition}
    \textbf{Digital signals} are sequences of numbers with quantized amplitudes.
\end{definition}
Here are some common signals you should be familiar with
\begin{itemize}
    \item unit impulse
    \item unit step
    \item sinusoid
    \begin{itemize}
        \item concept of period for discrete and continuous time
    \end{itemize}
\end{itemize}

\section{Systems Review}
In EE 120, we dealt mainly with LTI, linear time-invariant, systems. Let's define these terms.

\theoremstyle{definition}
\begin{definition}
    A \textbf{linear} system $T$ must satisfy $T\{x + y\} = T\{x\} + T\{y\}$ for any inputs $x, y$.
\end{definition}

\begin{definition}
    A \textbf{time-invariant} system $T$ must satisfy $T\{x(n - \tau)\} = T\{x\}(n - \tau)$ for any input $x$ and delay $\tau$.
\end{definition}

There are other important classifations of these systems as well.

\begin{definition}
    \textbf{Causal} sytems produce an output that is only dependent on current and prior values of the input.
\end{definition}
\begin{definition}
    \textbf{Memory-less} sytems produce an output that is only dependent on the current value of the input.
\end{definition}
\begin{definition}
    A signal $x$ is \textbf{bounded} if and only if we can define some finite value $c$ such that $\forall t, \abs{x(t)} \leq c$
\end{definition}
\begin{definition}
    A sytem is said to be \textbf{BIBO stable} if there does not exist any bounde input to the system that will produce an unbounded output.
\end{definition}

Let's look at some examples.

\begin{example}
    Consider the system $T\{x\}(n) = x(n - \tau)$. This system is LTI, causal, not memory-less, and BIBO stable.
\end{example}

\begin{example}
    Consider the system $T\{x\}(n) = \s{i}{-\infty}{n}{x(i)}$. This system is LTI, causal, not memory-less, and BIBO stable.
\end{example}

\begin{example}
    Consider the system $T\{x\}(n) = x[m \cdot t]$ for some scalar $m$. This system is linear, not time-invariant, not causal, not memory-less, and BIBO stable.
\end{example}

\chapter{120 Review Cont.}

\section{LTI Recap}
\subsection{Linearity and Time Invariance}
As discussed previously, we can prove linearity of a system by proving that it satisfies properties of superposition and scaling.

\theoremstyle{definition}
\begin{example}
    Consider a 3 point moving average filter, $y[n] = \frac{1}{3} (x[n - 1] + x[n] + x[n + 1])$. This represents a linear system.
\end{example}
\begin{example}
    Consider a 3 point median filter, $y[n] = \mathrm{Median}(x[n - 1], x[n], x[n + 1])$. This represents a non-linear system. Note that this non-linear filter still is used a lot in practice, since it's good at removing "shot noise".
\end{example}
Recall that we also need time-invariance to hold for a system to be LTI.
\subsection{Impulse Response}
\begin{definition}
    The \textbf{impulse response} of a system is the output of the system when the input is the unit impulse $\delta[n]$. We often label this output as $h[n]$.
\end{definition}
LTI systems are completely characterized by their impulse response.
\subsubsection{Output in terms of Impulse Reponse}
% Need to write convolution formula
% Need to write out proof
\begin{proof}
    Start
\end{proof}
\subsubsection{Stability in terms of Impulse Response}
\begin{theorem}
    BIBO stability in LTI Systems $\iff$ Impulse response is absolutely summable $\s{n}{-\infty}{\infty}{\abs{h[n]}}$
\end{theorem}

\subsubsection{Causality and Memoryless Properties}
We can also show that these properties of LTI systems can be determiend through analysis of the impulse response.

\subsection{Eigenfunctions of LTI systems}
Recall the concept of eigenvectors
% Show scaling by linear transformation
Similarly, we may define eigenfunctions of LTI systems to. In contrast to linear algebra, we may identify all eigenfucntions of an LTI system quite easily since they all take on the form $e^{j \omega n}$
\begin{definition}
    The \textbf{frequency response} of an LTI system is the function $H(e^{j \omega}) = \s{n}{-\infty}{\infty}{h[n] e^{-j \omega n}}$. This is also sometimes refered to as the \textbf{transfer function}.
\end{definition}
\begin{theorem}
    The output of an LTI system to input $e^{j \omega n}$ us $H(e^{j\omega}) e^{j\omega}$
\end{theorem}
\begin{proof}
    %Need to do this
\end{proof}
\section{Discrete-Time Fourier Transform (DTFT)}
Discussion of the frequency response now brings us to the DTFT. 
\begin{definition}
    The \textbf{discrete-time fourier transform} of a signal is $\mathrm{DTFT}\{x[n]\} = X(e^{j\omega}) = \s{n}{-\infty}{\infty}{x[n] e^{-j\omega n}}$. We can also define the \textbf{inverse discrete-time fourier transform} as $x[n] = \frac{1}{2 \pi} \I{\omega}{-\pi}{\pi}{X(e^{j\omega n})}$
\end{definition}
We can see how the DTFT definition is motivated by the definition of the frequency response from earlier.

The DTFT has a nice dual to the classical fourier series.
\begin{definition}
    %Put down fourier series formula
    The \textbf{fourier series} of a continuous signal is $f_{T}(t) = \s{-\infty}{\infty}{k}{}$
\end{definition}
% explain that the fourier series is just DTFT formula but swap frequency and time and period is 2 pi
 
\theoremstyle{definition}
\begin{example}
    Define a signal
    \begin{equation*}
        w[n] = \begin{cases}
            1 & 0 \leq n \leq N - 1 \\
            0 & \text{else}
        \end{cases}
    \end{equation*}
    %include picture
    This is a finite impulse train.
    We will compute the DTFT of this signal
    \begin{align*}
        W(e^{j\omega}) &= \s{k}{0}{N - 1}{e^{-j \omega k}} \\
        &= \s{k}{0}{N - 1}{\alpha^k} \\
        &= \frac{1 - \alpha^N}{1 - \alpha} \\
        &= \frac{1 - e^{-j \omega N}}{1 - e^{- j \omega}} \\
        &= \frac{e^{-j \omega \frac{N}{2}} [e^{-j \omega \frac{N}{2}} - e^{j \omega \frac{N}{2}}]}{} \\
        % more algebra...
        \Aboxed{W(e^{j \omega}) &= \abs{\frac{}{}}}
        %Fix this - you get a periodic sinc
    \end{align*}
\end{example}
\subsubsection{Properties}
\begin{itemize}
    \item Convolution in time is multiplication in frequency
    \begin{equation*}
        y[n] = (x * h)[n] \leftrightarrow Y(e^{j \omega n}) = H(e^{j \omega}) X(e^{j \omega})
    \end{equation*}
    \item Conjugate symmetry: If $x[n]$ is real...
    \begin{equation*}
        X^*(e^j \omega) = X(e^{- j \omega n})
    \end{equation*}
    We can actually exploit this redundancy in the requency specturm to save time and money
    \item Parseval's Theorem
    First, we state generalized Parseval's theorem
    \begin{equation*}
        \s{n}{-\infty}{\infty}{x[n] * y[n]} = \frac{1}{2 \pi} \I{\omega}{-\infty}{\infty}{X(e^{j \omega}) Y^*({e ^{- j \omega}})}
    \end{equation*}
    and then here is a form that shows how energy is preserved
    \item Shift property
\end{itemize}
\end{document}